% This is "sig-alternate.tex" V2.0 May 2012
% This file should be compiled with V2.5 of "sig-alternate.cls" May 2012
%
% This example file demonstrates the use of the 'sig-alternate.cls'
% V2.5 LaTeX2e document class file. It is for those submitting
% articles to ACM Conference Proceedings WHO DO NOT WISH TO
% STRICTLY ADHERE TO THE SIGS (PUBS-BOARD-ENDORSED) STYLE.
% The 'sig-alternate.cls' file will produce a similar-looking,
% albeit, 'tighter' paper resulting in, invariably, fewer pages.
%
% ----------------------------------------------------------------------------------------------------------------
% This .tex file (and associated .cls V2.5) produces:
%       1) The Permission Statement
%       2) The Conference (location) Info information
%       3) The Copyright Line with ACM data
%       4) NO page numbers
%
% as against the acm_proc_article-sp.cls file which
% DOES NOT produce 1) thru' 3) above.
%
% Using 'sig-alternate.cls' you have control, however, from within
% the source .tex file, over both the CopyrightYear
% (defaulted to 200X) and the ACM Copyright Data
% (defaulted to X-XXXXX-XX-X/XX/XX).
% e.g.
% \CopyrightYear{2007} will cause 2007 to appear in the copyright line.
% \crdata{0-12345-67-8/90/12} will cause 0-12345-67-8/90/12 to appear in the copyright line.
%
% ---------------------------------------------------------------------------------------------------------------
% This .tex source is an example which *does* use
% the .bib file (from which the .bbl file % is produced).
% REMEMBER HOWEVER: After having produced the .bbl file,
% and prior to final submission, you *NEED* to 'insert'
% your .bbl file into your source .tex file so as to provide
% ONE 'self-contained' source file.
%
% ================= IF YOU HAVE QUESTIONS =======================
% Questions regarding the SIGS styles, SIGS policies and
% procedures, Conferences etc. should be sent to
% Adrienne Griscti (griscti@acm.org)
%
% Technical questions _only_ to
% Gerald Murray (murray@hq.acm.org)
% ===============================================================
%
% For tracking purposes - this is V2.0 - May 2012

\documentclass{sig-alternate}

\begin{document}
%
% --- Author Metadata here ---
\conferenceinfo{WOODSTOCK}{'97 El Paso, Texas USA}
%\CopyrightYear{2007} % Allows default copyright year (20XX) to be over-ridden - IF NEED BE.
%\crdata{0-12345-67-8/90/01}  % Allows default copyright data (0-89791-88-6/97/05) to be over-ridden - IF NEED BE.
% --- End of Author Metadata ---

\title{Fighting Game AI Using The Finite State Machine Approach
\titlenote{(Produces the permission block, and
copyright information). For use with
SIG-ALTERNATE.CLS. Supported by ACM.}}
\subtitle{[Extended Abstract]
\titlenote{A full version of this paper is available as
\textit{Author's Guide to Preparing ACM SIG Proceedings Using
\LaTeX$2_\epsilon$\ and BibTeX} at
\texttt{www.acm.org/eaddress.htm}}}
%
% You need the command \numberofauthors to handle the 'placement
% and alignment' of the authors beneath the title.
%
% For aesthetic reasons, we recommend 'three authors at a time'
% i.e. three 'name/affiliation blocks' be placed beneath the title.
%
% NOTE: You are NOT restricted in how many 'rows' of
% "name/affiliations" may appear. We just ask that you restrict
% the number of 'columns' to three.
%
% Because of the available 'opening page real-estate'
% we ask you to refrain from putting more than six authors
% (two rows with three columns) beneath the article title.
% More than six makes the first-page appear very cluttered indeed.
%
% Use the \alignauthor commands to handle the names
% and affiliations for an 'aesthetic maximum' of six authors.
% Add names, affiliations, addresses for
% the seventh etc. author(s) as the argument for the
% \additionalauthors command.
% These 'additional authors' will be output/set for you
% without further effort on your part as the last section in
% the body of your article BEFORE References or any Appendices.

\numberofauthors{2} %  in this sample file, there are a *total*
% of EIGHT authors. SIX appear on the 'first-page' (for formatting
% reasons) and the remaining two appear in the \additionalauthors section.
%
\author{
% You can go ahead and credit any number of authors here,
% e.g. one 'row of three' or two rows (consisting of one row of three
% and a second row of one, two or three).
%
% The command \alignauthor (no curly braces needed) should
% precede each author name, affiliation/snail-mail address and
% e-mail address. Additionally, tag each line of
% affiliation/address with \affaddr, and tag the
% e-mail address with \email.
%
% 1st. author
\alignauthor
Xinwei Xiong\titlenote{Trinity College Dublin student}\\
       \affaddr{Trinity College Dublin}\\
       \affaddr{14307752}\\
       \email{xiongx@tcd.ie}
% 2nd. author
\alignauthor
Huanxiang Wang\titlenote{Trinity College Dublin student}\\
       \affaddr{Trinity College Dublin}\\
       \affaddr{14333168}\\
       \email{huwang@tcd.ie}
}
% There's nothing stopping you putting the seventh, eighth, etc.
% author on the opening page (as the 'third row') but we ask,
% for aesthetic reasons that you place these 'additional authors'
% in the \additional authors block, viz.

\date{26 Jan 2015}
% Just remember to make sure that the TOTAL number of authors
% is the number that will appear on the first page PLUS the
% number that will appear in the \additionalauthors section.

\maketitle
\begin{abstract}
Fighting Game AI Competition is an AI competition in which people develop AI controllers for Java based fighting game “FightingICE”. The competition and the platform are organized and maintained by Intelligent computer Entertainment Lab. Ritsumeikan University.  This platform allows you to design flexible AIs: you can create AI algorithms in order to use on NPCs. This means that you can make unique attacks, combos and movements, etc., depending on the algorithms in use; you can also make and learn action patterns from your opponent.

The developers have tried to include every imaginable sort
of ``bells and whistles", such as a subtitle, footnotes on
title, subtitle and authors, as well as in the text, and
every optional component (e.g. Acknowledgments, Additional
Authors, Appendices), not to mention examples of
equations, theorems, tables and figures.

To make best use of this sample document, run it through \LaTeX\
and BibTeX, and compare this source code with the printed
output produced by the dvi file. A compiled PDF version
is available on the web page to help you with the
`look and feel'.
\end{abstract}

% A category with the (minimum) three required fields
\category{H.4}{Information Systems Applications}{Miscellaneous}
%A category including the fourth, optional field follows...
\category{D.2.8}{Software Engineering}{Metrics}[complexity measures, performance measures]

\terms{Theory}

\keywords{ACM proceedings, \LaTeX, text tagging}

\section{Introduction}
The \textit{proceedings} are the records of a conference.
ACM seeks to give these conference by-products a uniform,
high-quality appearance.  To do this, ACM has some rigid
requirements for the format of the proceedings documents: there
is a specified format (balanced  double columns), a specified
set of fonts (Arial or Helvetica and Times Roman) in
certain specified sizes (for instance, 9 point for body copy),
a specified live area (18 $\times$ 23.5 cm [7" $\times$ 9.25"]) centered on
the page, specified size of margins (1.9 cm [0.75"]) top, (2.54 cm [1"]) bottom
and (1.9 cm [.75"]) left and right; specified column width
(8.45 cm [3.33"]) and gutter size (.83 cm [.33"]).

The good news is, with only a handful of manual
settings\footnote{Two of these, the {\texttt{\char'134 numberofauthors}}
and {\texttt{\char'134 alignauthor}} commands, you have
already used; another, {\texttt{\char'134 balancecolumns}}, will
be used in your very last run of \LaTeX\ to ensure
balanced column heights on the last page.}, the \LaTeX\ document
class file handles all of this for you.

The remainder of this document is concerned with showing, in
the context of an ``actual'' document, the \LaTeX\ commands
specifically available for denoting the structure of a
proceedings paper, rather than with giving rigorous descriptions
or explanations of such commands.

\section{Performance Evaluation}
We did the performance evaluation by matching our bot against all the competitors from the 3C category of 2014 competition, 3 rounds for each. Table 1 shows the results of the matching test.

\begin{table}
\centering
\begin{tabular}{|l|c|c|} 
\hline
Opponent & Average Score & Result \\ \hline
ATTeam2 & 320.3 : 678.7 & Lost \\ \hline
DragonKing3C & 820.7 : 178.3 & Win \\ \hline
LittleFuzzy & 611.3 : 387.7 & Win \\ \hline
PasanAI2 & 737 : 262.3 & Win \\ \hline
PnumaSON3C & 781.3 : 217.7 & Win \\ \hline
SealSwitch & 690.3 : 308.7 & Win \\ \hline
SejongFighter & 457.3 : 541.6 & Lost \\ \hline
T3c & 749 : 250 & Win \\ \hline
\end{tabular}
\caption{Results against All Competitors from 3C Category of 2014}
\end{table}

\begin{figure}[htp]
\begin{center}
\includegraphics[scale=0.25]{20150126120905.png}
\caption{A battle scenario}
\label{f10}
\end{center}
\end{figure}

As we can see from the table, our bot is good enough to defeat most of the competitors. This result indicates that the state machine designed for our bot is robust and comprehensive enough to deal with the most complicated battle environment. And this is one of the advantages benefited from the approach of finite state machine.

The ATTeam2 bot also used a state machine to control its behaviours. It has 3 basic states, defence, attack and counter attack. Their approach is very close to ours, but they must have a more detailed design than ours.

The SejongFighter has specialized tactics against different opponent's characters. Since different characters have different skills, attack range, etc., their approach is much more detailed and targeted.

\section{Difficulties Encountered}
We came across a series of difficulties when implementing the finite state machine into the actual bot.

\section{Assessment of Finite State Machine}
There are 3 use cases State Machines: Artificial Intelligence, modeling simulations, and a refactoring path for game entities.

For AI programming, you have the choice of deterministic or non-deterministic. Most video games have deterministic; meaning you know how the enemies will react based on different inputs. Let's take a typical spy game as an example. If the enemies don't see the player, they'll patrol. If they see the player, they'll attack him. If the player hides and waits, they'll eventually go back to patrolling. You can simulate fuzzy logic by using random numbers to change state to something random. For example, sometimes the enemy could use a grenade instead of a gun, or attack the player immediately instead of sounding the alarm, or even run away to get reinforcements. Sometimes they could even talk to the player instead. This randomness makes a State Machine non-deterministic, meaning you know all the States, but you don’t know exactly all the paths between States.


\section{Conclusions and Future Work}


%
% The following two commands are all you need in the
% initial runs of your .tex file to
% produce the bibliography for the citations in your paper.
\bibliographystyle{abbrv}
\bibliography{sigproc}  % sigproc.bib is the name of the Bibliography in this case
% You must have a proper ".bib" file
%  and remember to run:
% latex bibtex latex latex
% to resolve all references
%
% ACM needs 'a single self-contained file'!
%
%APPENDICES are optional
%\balancecolumns
\appendix
%Appendix A
\section{Headings in Appendices}
The rules about hierarchical headings discussed above for
the body of the article are different in the appendices.
In the \textbf{appendix} environment, the command
\textbf{section} is used to
indicate the start of each Appendix, with alphabetic order
designation (i.e. the first is A, the second B, etc.) and
a title (if you include one).  So, if you need
hierarchical structure
\textit{within} an Appendix, start with \textbf{subsection} as the
highest level. Here is an outline of the body of this
document in Appendix-appropriate form:
\subsection{Introduction}
\subsection{The Body of the Paper}
\subsubsection{Type Changes and  Special Characters}
\subsubsection{Math Equations}
\paragraph{Inline (In-text) Equations}
\paragraph{Display Equations}
\subsubsection{Citations}
\subsubsection{Tables}
\subsubsection{Figures}
\subsubsection{Theorem-like Constructs}
\subsubsection*{A Caveat for the \TeX\ Expert}
\subsection{Conclusions}
\subsection{Acknowledgments}
\subsection{Additional Authors}
This section is inserted by \LaTeX; you do not insert it.
You just add the names and information in the
\texttt{{\char'134}additionalauthors} command at the start
of the document.
\subsection{References}
Generated by bibtex from your ~.bib file.  Run latex,
then bibtex, then latex twice (to resolve references)
to create the ~.bbl file.  Insert that ~.bbl file into
the .tex source file and comment out
the command \texttt{{\char'134}thebibliography}.
% This next section command marks the start of
% Appendix B, and does not continue the present hierarchy
\section{More Help for the Hardy}
The sig-alternate.cls file itself is chock-full of succinct
and helpful comments.  If you consider yourself a moderately
experienced to expert user of \LaTeX, you may find reading
it useful but please remember not to change it.
%\balancecolumns % GM June 2007
% That's all folks!
\end{document}
